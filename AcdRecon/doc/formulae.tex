%%%%%%%%%% config.tex %%%%%%%%%%
%
% $Id$
%
\documentclass[fleqn,twoside,draft]{article}

%
% Dimensions
%
\textheight      9.0in               % Height of text part of page
\textwidth       7.0in               % Width of text part of page
\topmargin      -0.25in              % Nominal distance from top of page
\oddsidemargin  -0.25in              % Left margin on odd-numbered pages.
\evensidemargin -0.25in              % Left margin on even-numbered pages.

% 
% Common commands
%
\newcommand{\myIt}[1]{{\em #1}}
\newcommand{\register}[1]{{\em #1}}
\newcommand{\secref}[1]{sec. \ref{#1}}
\newcommand{\appendref}[1]{appendix \ref{#1}}
\newcommand{\Secref}[1]{Sec. \ref{#1}}
\newcommand{\Appendref}[1]{Appendix \ref{#1}}
\newcommand{\xmlNode}[1]{$<$#1$>$}
\newcommand{\executable}[1]{{\tt #1}}
\newcommand{\regMask}[1]{{\tt #1}}
%
% Packages
%
% To include postscript figures
\usepackage{graphicx}
% landscape tables
%\usepackage[figuresright]{rotating}
\usepackage{amsmath}

%
% Declarations for front matter
%
\title{Formuale for ACD Reconstruction}

\author{E. Charles}
             
\begin{document}

% typeset front matter
\maketitle


%
% Abstract
%
\begin{abstract}

This document describes various formulae used in the ACD reconstruction.
\vspace{1pc}
\end{abstract}

%
% Text proper
%
\section{Introduction \label{sec:introduction}}

All the ACD geometrical calculations are trying to establish answers to a simple question: {\em How close did a given track come to the edge a particular tile or ribbon?}.  The complications arise because of distinctions between 2D and 3D distances of closest approach as questions as to where the track parameters are defined.  

The ACD reconstruction uses three different types of calculations.
\begin{enumerate}
  {\item Finding the point of closest approach to a point.}
  {\item Finding the point of closest approach to a ray.}
  {\item Projecting a track to a plane.}
\end{enumerate}

In the first two cases the relevant quantities are the point of closest approach (POCA), the vector of closest approach (VOCA) the distance of closest approach (DOCA), and the projection of the track covariance matrix along the vector of closest approach, which is also the estimate of the error on the DOCA.  In the last case the relevant quantities are the intersection point, the distances along the plane to the edges of the tile, and projection of the track covariance matrix onto the plane.  
%
% Section on Formualae
%   Sub-sections: Trigger, General, ACD, CAL, TKR
% 
\section{Formulae \label{sec:formulae}}

\subsection{Conventions}

\subsubsection{Tracks}

The TKR Kalman fit expresses the track parameters and covariance matrix in terms of the slope and intercept in each projection $(x_0,s_x,y_0,s_y)$ where the free parameter is $z$.  Since some of the ACD tiles are not oriented in X-Y plane it is much easier to work in terms of a track express as vector and a reference point $(v_x,v_y,v_z,x_0,y_0)$ with the free parameter being the path-length along the track $s$.  The transformation equations are:

\begin{subequations}
  \begin{gather}
    \hat{v}_x = \frac{s_x}{(1 + {s_x}^2 + {s_y}^2)^{1/2}} \\
    \hat{v}_y = \frac{s_y}{(1 + {s_x}^2 + {s_y}^2)^{1/2}} \\
    \hat{v}_z = \frac{1}{(1 + {s_x}^2 + {s_y}^2)^{1/2}} \\
    x_0 = x_0 \\
    y_0 = y_0 
  \end{gather}
\end{subequations}

The corresponding covariance $5\times4$ derivative matrix is:

\begin{equation}
  A = 
  \begin{pmatrix}
    0 & \hat{v}_z ( 1 - { \hat{v}_x } ^2 ) & 0 & -\hat{v}_x\hat{v}_y\hat{v}_z \\
    0 & -\hat{v}_x\hat{v}_y\hat{v}_z & 0 & \hat{v}_z ( 1 - { \hat{v}_y } ^2 ) \\
    0 & -\hat{v}_x{\hat{v}_z}^2 & 0 & -\hat{v}_y{\hat{v}_z}^2\\
    1 & 0 & 0 & 0 \\
    0 & 0 & 1 & 0 
  \end{pmatrix}
\end{equation}

Given these equations we can express an track parameters in the form $(\hat{v},\vec{x})$ and transform the covariance matrix from the $4\times4$ slope-intercept (${\sigma_{x,s}}^2$) to the $5\times5$ vector-point form (${\sigma_{v,x}}^2$):
\begin{equation}
  (\sigma_{v,x})^2 = A {\sigma_{v,x}}^2 A^T
\end{equation}

It is also useful to write down esplicitly the inverse transformation, which turns out to be handy when one needs to convert to the tracker representation a track which is espressed in the $(\hat{v},\vec{x})$. This is needed, for instance, if one wants to Kalman propagate to the ACD a calorimeter track (i.e. the direction of a calorimeter cluster).
The transformation equations read:
\begin{subequations}
  \begin{gather}
   x_0 = x_0\\
    s_x = \frac{\hat{v}_x}{\hat{v}_z}\\
    y_0 = y_0\\
    s_y = \frac{\hat{v}_y}{\hat{v}_z}
  \end{gather}
\end{subequations}
and the corresponding $4\times5$ covariance matrix is:
\begin{equation}
  B = 
  \begin{pmatrix}
   0  & 0 & 0 & 1 & 0\\
   1/\hat{v}_z  & 0 & -\hat{v}_x/\hat{v}_z^2 & 0 & 0\\
   0  & 0 & 0 & 0 & 1\\
   0  & 1/\hat{v}_z & -\hat{v}_y/\hat{v}_z^2 & 0 & 0\\
  \end{pmatrix}
    =
  \begin{pmatrix}
   0  & 0 & 0 & 1 & 0\\
   (1 + s_x^2 + s_y^2)^{1/2}  & 0 & -s_x(1 + s_x^2 + s_y^2)^{1/2} & 0 & 0\\
    0 & 0 & 0 & 0 & 1\\
    0 & (1 + s_x^2 + s_y^2)^{1/2} & -s_y(1 + s_x^2 + s_y^2)^{1/2} & 0 & 0\\
  \end{pmatrix}
\end{equation}
We wrote the last matrix in terms of both the old and the new variables in such a way it is immediate to verify that:
\begin{equation}
  BA = 
  \begin{pmatrix}
   0  & 0 & 0 & 1 & 0\\
   1/\hat{v}_z  & 0 & -\hat{v}_x/\hat{v}_z^2 & 0 & 0\\
   0  & 0 & 0 & 0 & 1\\
   0  & 1/\hat{v}_z & -\hat{v}_y/\hat{v}_z^2 & 0 & 0\\
  \end{pmatrix}
  \begin{pmatrix}
    0 & \hat{v}_z ( 1 - { \hat{v}_x } ^2 ) & 0 & -\hat{v}_x\hat{v}_y\hat{v}_z \\
    0 & -\hat{v}_x\hat{v}_y\hat{v}_z & 0 & \hat{v}_z ( 1 - { \hat{v}_y } ^2 ) \\
    0 & -\hat{v}_x{\hat{v}_z}^2 & 0 & -\hat{v}_y{\hat{v}_z}^2\\
    1 & 0 & 0 & 0 \\
    0 & 0 & 1 & 0 
  \end{pmatrix}
  = {\rm I_{4\times4}}
\end{equation}
On a side, we note that the trick in the other verse does not work. The $5\times5$ matrix $BA$ is not ${\rm I_{5\times5}}$, which apparently is a well known feature of the Jacobian matrices involving a change of dimensionality (the mapping from a 5-dimensional space to a 4-dimensional one is not unique).


\subsubsection{ACD Tiles}

For the most part ACD tiles are defined as flat planes.  A total of 10 tile on the sides of the top of the ACD which are bent to cover gaps in the ACD are approximated as two orthogonal planes.  In either case all the geometrical calculations are made with respect to the tile planes. 

Any given plane is defined in terms of a center point $\vec{q}$ and orientation matrix $R$.  The rows of the orientation matrix define unit vectors along the tile local $x,y,z$ axes, ($\hat{r}_{x},\hat{r}_{y},\hat{r}_{z}$).  The local axes are oriented so that the tile extends in $x,y$ and the the $z$ axis points out of the LAT.  Sometime we refer the normal to the plane $\hat{r}_{z}$ as the normal vector $\hat{n}$.  Finally, the active region of the plane is defined by the extent of the tile in local $x,y$.  

For three dimensional calculations we use the corners of the tile $\vec{c}_i$.  We also define the 4 rays that connect each set of adjacent corners.  Each ray is specified by a direction vector and a start point $(\hat{r}_i,\vec{q}_i)$.  Note that $\hat{r}_i = \vec{c}_i - \vec{c}_{i+1}$ and $\vec{q}_i = \vec{c}_i$.

For tracks that pass inside the tile it is sufficient to find with edge ray the track comes closest to.  However, for tracks that miss the tile we must also consider all four corners, as the point of closest approach might not fall inside any of the tile edges.

\subsubsection{ACD Ribbons}

The ACD ribbons are modeled as a series of rays with a direction and a start point $(\hat{r}_i,\vec{q}_i)$.  

For each of the ribbon rays we check if the POCA falls along that ray.  If it does not we must also consider the POCA at the relevant end of the ray.  The POCA w.r.t. the entire ribbon is the closest POCA of any of the various segments.


\subsection{POCA Between a Track and a Point}

Given a track defined by $(\hat{v},\vec{x})$ with a corresponding covariance matrix $(\sigma_{v,x})^2$ and a point $\vec{q}$, the POCA ($\vec{p}$), VOCA ($\vec{J}$) and DOCA ($d$) are:

\begin{subequations}
\begin{gather}
  \vec{p} = \vec{x} + s \hat{v} \\
  \vec{J} = \vec{\Delta} + s \hat{v} \\
  d = |\vec{J}| 
\end{gather}
\end{subequations}

Where $\vec{\Delta}$ is the vector between track and ray reference points: $\vec{\Delta} = \vec{x} - \vec{q}$.  Furthermore, $s$ is the path-length along the track where the POCA occurs.  It is given by:
\begin{equation}  
  s = \vec{\Delta} \cdot \hat{v}
\end{equation}

The derivatives to transform the covariance matrix from the $(\vec{v},\vec{x})$ representation to the errors on $\vec{p}$ are:
\begin{subequations}
\begin{gather}
  \frac{\partial J_{i}}{\partial v_{j}} = \Delta_i \hat{v}_{j} + s \delta_{ij} \\
  \frac{\partial J_{i}}{\partial x_{j}} = \hat{v}_{i} \hat{v}_{j} + \delta_{ij} \\
\end{gather}
\end{subequations}

Given these equations we can work out the 3 X 5 derivative matrix $B_{(point)}$ as:

\begin{equation}
  B_{(point)} = 
\begin{pmatrix}
  \Delta_0 \hat{v}_{0} + s & 
  \Delta_0 \hat{v}_{1} & 
  \Delta_0 \hat{v}_{2} & 
  \hat{v}_{0} \hat{v}_{0} + 1 & 
  \hat{v}_{0} \hat{v}_{1} \\
  \Delta_1 \hat{v}_{0} & 
  \Delta_0 \hat{v}_{1} + s & 
  \Delta \hat{v}_{2} & 
  \hat{v}_{1} \hat{v}_{0} & 
  \hat{v}_{1} \hat{v}_{1} + 1 \\
  \Delta_2 \hat{v}_{0} & 
  \Delta_1 \hat{v}_{1} & 
  \Delta \hat{v}_{2} + s &
  \hat{v}_{2} \hat{v}_{0} &
  \hat{v}_{2} \hat{v}_{1}
\end{pmatrix}
\end{equation}

Then the errors on the vector of closest approach (${\sigma_{J}}^2$) and the distance of closest approach (${\sigma_{d}}^2$) are given by:

\begin{subequations}
\begin{gather}
  {\sigma_{J}}^2 =  B {\sigma_{v,x}}^2 B^{T} \\
  {\sigma_{d}}^2 =  J {\sigma_{J}}^2 p^T = p B {\sigma_{v,x}}^2 B^{T} J^T 
\end{gather}
\end{subequations}



\subsection{POCA Between a Track and a Ray}

Given a track defined by $(\hat{v},\vec{x})$ with a corresponding covariance matrix $({\sigma_{v,x}}^2)$ and a ray defined by $(\hat{r},\vec{q})$, the track POCA $(\vec{p})$, VOCA $(\vec{J})$ and DOCA $(d)$ are given by:

\begin{subequations}
\begin{gather}
  \vec{p} = \vec{x} + s \hat{v} \\
  \vec{J} = \vec{\Delta} + s \hat{v} - t \hat{r} \\
  d = |\vec{J}| 
\end{gather}
\end{subequations}

Where $\vec{\Delta}$ is the vector between track and ray reference points: $\vec{\Delta} = \vec{x} - \vec{q}$.  Furthermore, $s$ and $t$ are respectively the path-lengths along the track and the ray at with the POCA occurs.  They can be expressed in terms of the dot product between the track and ray directions $b = \hat{v} \cdot \hat{r}$.  The formulae are:

\begin{subequations}
\begin{gather}
  s = \frac{ b (\hat{r} \cdot \vec{\Delta}) - (\hat{v} \cdot \vec{\Delta})}{ 1 - b^2 } \\
  t = \frac{ (\hat{r} \cdot \vec{\Delta}) - b (\hat{v} \cdot \vec{\Delta})}{ 1 - b^2 } 
\end{gather}
\end{subequations}

The derivatives to transform the covariance matrix from the $(\vec{v},\vec{x})$ representation to the errors on $\vec{J}$ are:
\begin{subequations}
\begin{gather}
  \frac{\partial J_{i}}{\partial v_{j}} = \frac{ \alpha_{0i} \hat{v}_j + \alpha_{1i} \hat{r}_j}{ 1 - b^2 } 
                                        + s \delta_{ij} \\
  \frac{\partial J_{i}}{\partial x_{j}} = \frac{ \beta_{0i} \hat{v}_j - \beta{_1i} \hat{r}_j}{ 1 - b^2 } 
                                        + \delta_{ij} 
\end{gather}
\end{subequations}

Where $\vec{\alpha}_0, \vec{\alpha}_1, \vec{\beta}_0, \vec{\beta}_1$ are various useful vectors defined by:

\begin{subequations}
\begin{gather}
  \vec{\alpha}_0 = (\hat{r} \cdot \vec{\Delta} + 2bs) \hat{r} - \vec{\Delta} \\
  \vec{\alpha}_1 = (\hat{v} \cdot \vec{\Delta} + 2bt) \hat{r} - b\vec{\Delta} \\
  \vec{\beta}_0 =  b\hat{r} + \hat{v} \\
  \vec{\beta}_1 =  \hat{r} + b\hat{v}
\end{gather}
\end{subequations}

Given these equations we can work out the 3 x 5 derivative matrix $B_{(ray)}$ as:

\begin{equation}
  B_{(ray)} = 
\begin{pmatrix}
    \frac{ \alpha_{00} \hat{v}_0 + \alpha_{10} \hat{r}_0}{ 1 - b^2 } + s & 
    \frac{ \alpha_{00} \hat{v}_1 + \alpha_{10} \hat{r}_1}{ 1 - b^2 } &
    \frac{ \alpha_{00} \hat{v}_2 + \alpha_{10} \hat{r}_2}{ 1 - b^2 } &
    \frac{ \beta_{00} \hat{v}_0 - \beta{_10} \hat{r}_0}{ 1 - b^2 } + 1 &
    \frac{ \beta_{00} \hat{v}_1 - \beta{_10} \hat{r}_1}{ 1 - b^2 } \\
    \frac{ \alpha_{01} \hat{v}_0 + \alpha_{11} \hat{r}_0}{ 1 - b^2 } & 
    \frac{ \alpha_{01} \hat{v}_1 + \alpha_{11} \hat{r}_1}{ 1 - b^2 } + s &
    \frac{ \alpha_{01} \hat{v}_2 + \alpha_{11} \hat{r}_2}{ 1 - b^2 } &
    \frac{ \beta_{01} \hat{v}_0 - \beta{_11} \hat{r}_0}{ 1 - b^2 } &
    \frac{ \beta_{01} \hat{v}_1 - \beta{_11} \hat{r}_1}{ 1 - b^2 } + 1 \\
    \frac{ \alpha_{02} \hat{v}_0 + \alpha_{12} \hat{r}_0}{ 1 - b^2 } & 
    \frac{ \alpha_{02} \hat{v}_1 + \alpha_{12} \hat{r}_1}{ 1 - b^2 } &
    \frac{ \alpha_{02} \hat{v}_2 + \alpha_{12} \hat{r}_2}{ 1 - b^2 } + s &
    \frac{ \beta_{02} \hat{v}_0 - \beta{_12} \hat{r}_0}{ 1 - b^2 } &
    \frac{ \beta_{02} \hat{v}_1 - \beta{_12} \hat{r}_1}{ 1 - b^2 }
\end{pmatrix}
\end{equation}

Then the errors on the vector of closest approach (${\sigma_{J}}^2$) and the distance of closest approach (${\sigma_{d}}^2$) are given by:

\begin{subequations}
\begin{gather}
  {\sigma_{J}}^2 =  B {\sigma_{v,x}}^2 B^T \\
  {\sigma_{d}}^2 =  J {\sigma_{J}}^2 J^T = J B {\sigma_{v,x}}^2 B^T J^T
\end{gather}
\end{subequations}

\subsection{POCA Between a Track and a Plane}

This section is largely adapted from R. Johnson's earlier note \cite{planeError}.  Given a track defined by $(\hat{v},\vec{x})$ with a corresponding covariance matrix $({\sigma_{v,x}}^2)$ and a plane defined by a point ($\vec{q}$) and orientation matrix ($R = \hat{r}_x,\hat{r}_y,\hat{r}_z$, with $\hat{n} = \hat{r}_z$ normal to the plane), the track crosses the plane at a point the global frame $(\vec{p})$ and in the local frame of the place defined $(\vec{l})$  by:

\begin{subequations}
\begin{gather}
  \vec{p} = \vec{x} + s \hat{v} \\
  \vec{l} = R \vec{p} + \vec{q}
\end{gather}
\end{subequations}

Where the path-length along the track of the crossing point ($s$) can be calculated in terms of the vector between the reference points ($\vec{\Delta} = \vec{x} - \vec{q}$) and the plane normal vector $\hat{n}$.  The formula is:

\begin{equation}
  s = \frac{\vec{\Delta} \cdot \hat{n}}{ \hat{v} \cdot \hat{n}}
\end{equation}

The derivatives to transform the covariance matrix from the $(\vec{v},\vec{x})$ representation to the errors on $\vec{p}$ are:
\begin{subequations}
\begin{gather}
  \frac{\partial p_{i}}{\partial v_{j}} = \frac{- \hat{n}_j}{\hat{v}\cdot\hat{n}} \hat{v}_i
                                        + s \delta_{ij} \\
  \frac{\partial p_{i}}{\partial x_{j}} = -\frac{\Delta\cdot\hat{n}}{(\hat{v}\cdot\hat{n})^2}\hat{n}_{j}\hat{v}_i
                                        + \delta_{ij} 
\end{gather}
\end{subequations}

Given these equations we can work out the 3 X 5 derivative matrix $B_{(plane)}$ as:

\begin{equation}
  B_{(plane)} = 
\begin{pmatrix}
  \frac{- \hat{n}_0}{\hat{v}\cdot\hat{n}}\hat{v}_0 + s &
  \frac{- \hat{n}_1}{\hat{v}\cdot\hat{n}}\hat{v}_0 &
  \frac{- \hat{n}_2}{\hat{v}\cdot\hat{n}}\hat{v}_0 &
  1 - \frac{\Delta\cdot\hat{n}}{(\hat{v}\cdot\hat{n})^2}\hat{n}_0\hat{v}_0 &
  \frac{\Delta\cdot\hat{n}}{(\hat{v}\cdot\hat{n})^2}\hat{n}_1\hat{v}_0 \\
  \frac{- \hat{n}_0}{\hat{v}\cdot\hat{n}}\hat{v}_1 &
  \frac{- \hat{n}_1}{\hat{v}\cdot\hat{n}}\hat{v}_1 + s &
  \frac{- \hat{n}_2}{\hat{v}\cdot\hat{n}}\hat{v}_1 &
  - \frac{\Delta\cdot\hat{n}}{(\hat{v}\cdot\hat{n})^2}\hat{n}_0\hat{v}_0 &
  1 - \frac{\Delta\cdot\hat{n}}{(\hat{v}\cdot\hat{n})^2}\hat{n}_1\hat{v}_0 \\
  \frac{- \hat{n}_0}{\hat{v}\cdot\hat{n}}\hat{v}_2 &
  \frac{- \hat{n}_1}{\hat{v}\cdot\hat{n}}\hat{v}_2 &
  \frac{- \hat{n}_2}{\hat{v}\cdot\hat{n}}\hat{v}_2 + s&
  - \frac{\Delta\cdot\hat{n}}{(\hat{v}\cdot\hat{n})^2}\hat{n}_0\hat{v}_0 &
  - \frac{\Delta\cdot\hat{n}}{(\hat{v}\cdot\hat{n})^2}\hat{n}_1\hat{v}_0   
\end{pmatrix}
\end{equation}

Then the errors on the intersection point (${\sigma_{p}}^2$) and the projection into the plane (${\sigma_{l}}^2$) are given by:

\begin{subequations}
\begin{gather}
  {\sigma_{p}}^2 =  B {\sigma_{v,x}}^2 B^T \\
  {\sigma_{l}}^2 =  R {\sigma_{p}}^2 R^T = R B {\sigma_{v,x}}^2 B^T R^T
\end{gather}
\end{subequations}

\clearpage

%
% End of Main body.
%
%

%
% appendices.  
%   sections:  Registers, ACD Maps in GEM, LPA votes syntax. LCI votes syntax.  Ancillary Files syntax.
%
\appendix


\clearpage

\begin{thebibliography}{9}

%\cite{planeError}
\bibitem{planeError}
R. Johnson
``Projection of the Track Covariance Matrix onto an Arbitrary Plane''

\end{thebibliography}

\end{document}
