\documentclass[12pt,final]{article}

\usepackage{graphicx}
\usepackage{amsmath}

\begin{document}

\title{An Introduction to the Kalman Vertexer for GLAST}
\author{Johann Cohen-Tanugi}
\maketitle
%%%%%%%%%%%%%%%%%%%%%%%%%%%%%%%%%%%%%%%%%%
{\bf\huge The Kalman vertexer is still under construction, as well as this document! } 


\section{Miscellaneous preliminary remarks}

With respect to collider experiments, GLAST presents several peculiarities:
\begin{itemize}
\item tracks are straight lines, so that in principle a vertexing scheme should be much simplified, 
as track parameters evolution along the trajectory are trivial to compute (except maybe for the energy).
\item despite the ``granularity'' of the calorimeter, separate track by track 
energy determination is hard. Thus, one cannot put much hope in the use of secondary vertices 
to ``recover'' from delta rays, etc...
\item There is no ``beam spot'', and as a result no obvious initial vertex location.
The latter must be determined event by event. 
\item The discretized nature of the hits position measurements is expected to limit 
vertex determination accuracy in a non trivial way. It is also leading to an unusual
definition of track parameters, following track reconstruction paradigm (see below).
\end{itemize}

It is important to emphasize the fact that GLAST program relies on the measurements of 
incident gammas energy and direction. The position of the pair production vertex is not mandatory.
It may help to analyse an event, for instance by providing an ``in converter decay'' constraint. 
The Kalman vertexer will eventually provide for such capability.
Nevertheless, the main purpose of the Kalman vertexer lies in the fact that 
it provides an automated rigorous agregation of the different tracks covariance matrix. 
This is essential for a good understanding of the reconstruction errors, that would not rely on 
the MonteCarlo exclusively.


I note also here an interesting feature that I came aware of during a discussion with Bill Atwood:
Given the fact that track direction errors arise primarily from MS, which is independant for each
track, one can consider that each reconstructed track, provided that it is not too badly 
reconstructed, is an independent estimation of the initial gamma direction, to be properly weighted
by its error matrix. This gives rise to a prescription for gamma direction which is completely 
different from the case where one adds the physical momentum of the tracks to make up the candidate
photon momentum. The first prescription exemplified in TkrComboVtxRecon.cxx, amounts to the 
following estimation:
\begin{equation}
\begin{aligned}
dir_\gamma &= \frac{\frac{dir_{t_1}}{C_1}+\frac{dir_{t_2}}{C_2}}{\frac{1}{C_1}+\frac{1}{C_2}} 
\quad\text{symbolic notation}\\
\frac{1}{C(dir_\gamma)} &= \frac{1}{C(dir_{t_1})} + \frac{1}{C(dir_{t_2})}
\end{aligned}
\end{equation}

\noindent whereas the logical outcome of the Kalman vertexer will be
\footnote{The sum of momenta can actually not be equal to the initial momentum, as the nucleus must
have an unknown recoil. This should be completely negligible}

\begin{equation}
\begin{aligned}
p_\gamma &= p_{t_1} + p_{t_2} \quad\text{where $p$ must be the physical momentum}\\
C_\gamma &= C_1 + C_2 + Cov(1,2) + Cov(2,1)
\end{aligned}
\end{equation}


\section{Kalman Vertexer: general equations}
Given a list of tracks $k$, with their ``measured'' parameter 
vector $m_k$, a Kalman filter can be designed in order to estimate the common
vertex of tracks. It is an iterative process: starting with an initial estimate of the vertex $x^0$
and its covariance matrix $C^0$, it adds one track at a time and reestimate vertex and cov. matrix 
at each step (filter procedure.) At the end, a smoother procedure propagate the final result to 
all tracks in order to reestimate their parameters at the final vertex estimate.

Let's define the following:
\begin{itemize}
\item $x_k$: vertex estimate after using k tracks;
\item $x^t$: true vertex;
\item $\htmlimage{}C_k\equiv cov(x_k-x^t)$;
\item $q_k$: estimate of ``geometrical'' momentum of track $k$ at $x_k$.  ``geometrical''  is to be opposed to 
``physical'': it is somehow related to the local direction of a track, but doesn't need 
to be $p$ of $(E,p)$ 4-vector.
\item   $q^t_k$: true ``geometrical'' momentum of track $k$ at $x_k$;
\item $\htmlimage{}D_k\equiv cov(q_k - q^t_k)$;
\item $\htmlimage{}E_k\equiv cov((x_k-x^t) , (q_k-q^t_k))$;
\item $m_k$: parameter vector of track $k$;
\item $v_k$: random noise on  $m_k$, assumed to be gaussian;
\item $\htmlimage{}V_k\equiv cov(v_k)$. $\htmlimage{}G_k\equiv V_k^{-1}$ is the weight matrix of track $k$. 
\end{itemize}

Then, the Kalman procedure relies on the measurement equation, which defines a mapping from $m_k$
 to the $(x^t,q^t_k)$:

\begin{equation}
\begin{aligned}
m_k &= h_k(x^t,q^t_k) + v_k.
\end{aligned}
\end{equation}

Besides, its application requires $h_k$ to be linear. For $(x^0_k,q^0_k)$ chosen in the 
vicinity of $(x^t,q^t_k)$, linearization of $h_k$ yields:

\begin{equation}
\begin{aligned}
h_k(x^t,q^t_k) &\approx c^0_k + A_k x^t + B_k q^t \\
\text{where } A_k &= \frac{\partial h_k}{\partial x^t}\biggl\lvert_{(x^0_k,q^0_k)}\biggr.\\
              B_k &= \frac{\partial h_k}{\partial q^t_k}\biggl\lvert_{(x^0_k,q^0_k)}\biggr.\\
\text{and } c^0_k &= h_k(x^0_k,q^0_k) - A_k x^0_k- B_k q^0_k
\end{aligned}
\end{equation}

Usually, $x^0_k$ is set to $x_{k-1}$, {\it i.e.} the vertex estimate after $k-1$ step, and for $q^0_k$ 
the momentum of the track at the POCA to $x^0_k$.

A least $\chi_2$ prescription allows to update information at each step, with the following filtering formula:

\begin{equation}
\begin{aligned}
x_k &= C_k[C_{k-1}^{-1} x_{k-1} + A^T_k G^B_k(m_k-c^0_k)]\\
q_k &= W_k B^T_k G_k (m_k - c^0_k  -  A_k x_k)\\
C_k &= (C_{k-1}^{-1} + A^T_k G^B_k A_k)^{-1}\\
D_k &= W_k + W_k B^T_k G_k A_k C_k A^T_k G_k B_k W_k\\
E_k &= -C_k A^T_k G_k B_k W_k\\
W_k &= (C_{k-1}^{-1} + A^T_k G^B_k A_k)^{-1}\\
G^B_k &= G_k - G_k B_k W_k B^T_k G_k 
\end{aligned}
\end{equation}

The $\chi^2$ increase when adding track $k$ is computed as:

\begin{equation}
\begin{aligned}
\chi^2_{KF} &= r^T_k G_k r_k + (x_k - x_{k-1})^T C_{k-1}^{-1} (x_k - x_{k-1})\\
        r_k &= m_k - c^0_k - A_k x_k - B_k q_k.
\end{aligned}
\end{equation}

If $\chi^2_{KF}$ exceeds a user defined upper value, track $k$ can be discarded from 
the fit. The total $\chi^2$ is $\htmlimage{}\displaystyle{\chi^2_k = \chi^2_{k-1} + \chi^2_{KF}}$.

The smoothing step goes over all accepted tracks and recomputes $q^N_k$,$D^N_k$ and $E^N_k$, using
the final estimates $x$ and $C$ in the filtering equations instead of $x_k$ and $C_k$. It also 
allows to compute:
\begin{equation}
\begin{aligned}
Q_{kj} &\equiv cov(q^N_k-q^t_k,q^N_j-q^t_j) = W_k B^T_k G_k A_k C A^T_j G_j B_j W_j.
\end{aligned}
\end{equation}

Note that, in the event that $x$ is noticeably different from $x^0$, the whole procedure
can be performed again with $x^0$ and $C^0$ set to $x$ and $C$ respectively.


\section{Implementation in GLAST software }

GLAST Track fitting uses the following parametrization (one needs a point on the track, and the latter's direction):
\begin{itemize}
\item Track initial direction is given by projected slopes estimated at the first hit position along the track.
If $\htmlimage{}(u_x,u_y,u_z)$ is a unit vector along the track direction, projected slopes are defined by 
$\htmlimage{}\displaystyle{S_x=\frac{u_x}{u_z}}$ and  $\htmlimage{}\displaystyle{S_y=\frac{u_y}{u_z}}$.
\item The first hit location $(X,Y,Z)$ is used to complete the parametrization.
\item The $Z$ coordinate depends on the discrete localization of the first hit: it is decorrelated from 
$(X,S_X,Y,S_Y)$ which are estimated by a Kalman track fitting algorithm, along with their covariance matrix. 
The error on $Z$ is a matter of debate. We currently use $\dfrac{0.4}{\sqrt(12)}$mm, that is to say a uniform 
distribution of $Z$ between $X$- and $Y$-planes of first hit.
\end{itemize}

In order to remain consistent with the track reconstruction paradigm, we use projected slopes in order to
parametrize the direction of the track: by definition, the ``geometrical'' momentum is the vector $(S_X,S_Y,E)$ 
defined for each track.
The track being a straight line, it doesn't change all along the track
\footnote{We assume that $E$ doesn't need to be reevaluated.}, so that there is no extra work required to
compute $q^0_k$ from the measured $q_k$.

In order to determine the measurement equation, one needs to define a reference plane where ``geometrical'' 
momentum is defined. In collider experiment, this would be, typically, the normal plane to the beamline, 
containing the POCA on the track to the beamline ({\it i.e.} Z=0 axis).
 
In the case of GLAST, we take as reference plane the horizontal plane containing the first hit of
the best track, which is the first one in the track list. 
This seems a reasonable, though probably biasing, estimate for a vicinity to the true vertex.
All tracks should have their measurement vector $m_k$ defined at this point
\footnote{as well as the covariance matrix, which is the only ``extra work'' needed.}.
As a result, the measurement equation (without noise) $\htmlimage{}m_k = h_k(x^t,q^t_k)$ reads:
\begin{equation}
\begin{aligned}
X_k      &= X^t + S^t_{X}(Z_{ref} - Z^t)\\
{S_X}_k  &= S^t_X\\
Y_k      &= Y^t + S^t_Y(Z_{ref} - Z^t)\\
{S_Y}_k  &= S^t_Y\\
E_k      &= E^t_k
\end{aligned}
\end{equation}

Matrices $A_k$ and $B_k$ are straightforward to compute:
\begin{equation}
\begin{aligned}
A_k &=
\begin{pmatrix}
1 & 0 & -{S_X}^0_k\\
0 & 0 & 0\\
0 & 1 & -{S_Y}^0_k\\
0 & 0 & 0\\
0 & 0 & 0\\
\end{pmatrix}
\quad\quad
B_k &=
\begin{pmatrix}
Z_{ref} - Z^0_k & 0                & 0\\
1               & 0                & 0\\
0               & Z_{ref} - Z^0_k  & 0\\
0               & 1                & 0\\
0               & 0                & 1\\
\end{pmatrix}
\end{aligned}
\end{equation}

It is interesting to note that the measurement equation is very close to being linear:
\begin{equation}
\begin{aligned}
h_k(x^t,q^t_k) &= h_k(x^0_k,q^0_k) + A_k (x^t-x^0_k) + B_k (q^t_k-q^0_k) -
\begin{pmatrix}
(S^t_X-{S_X}^0_k)(z^t-z^0_k)\\
0\\
(S^t_Y-{S_Y}^0_k)(z^t-z^0_k)\\
0\\
0
\end{pmatrix}
\end{aligned}
\end{equation}

\section{On the change of parameters}
Once it is computed, the total covariance matrix must be reformulated in the same parameters as
the ``physical momentum'', which needs a transformation $\htmlimage{}(S_X,S_Y, E)\rightarrow(Eu_x,Eu_y,Eu_z)$.


\begin{equation}
\begin{aligned}
T &= \frac{\partial(Eu_x,Eu_y,Eu_z)}{(S_X,S_Y,E)} \\
&= \frac{sign(u_z)}{(1+S_X^2+S_Y^2)^{\frac{3}{2}}}
\begin{pmatrix}
E(1+S_Y^2)     & -ES_XS_Y     & S_X(1+S_X^2+S_Y^2) \\
-ES_XS_Y       & E(1+S_X^2)   & S_Y(1+S_X^2+S_Y^2) \\
-ES_X          & -ES_Y        & 1+S_X^2+S_Y^2
\end{pmatrix}
\end{aligned}
\end{equation}

\noindent and the transformation formula: $\displaystyle{Q' = T^T\cdot Q\cdot T}$.




\section{caveat and todo list}
caveats:
\begin{itemize}
\item No iteration in case of a clear shift of the vertex position;
\item No propagation of the Covariance matrix yet;
\item Energy integration is ``fake''. Will we be able to do better?
\end{itemize}
To do List:
\begin{itemize}
\item Still systematic studies needed to estimate its viability: on the top of the task list.
\item Separation of general tools/utilities (parameter transformation, cov. propagation, etc...)
\item To really study this vertexer, we need to enlarge TDS/PDS! 
\end{itemize}
%%%%%%%%%%%%%%%%%%%%%%%%%%%%%%%%%%%%%%%%%%

\end{document}
